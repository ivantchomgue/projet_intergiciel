\documentclass[t, usepdftitle=false]{beamer}% t=top, b=bottom, c=center(centré verticalement pas horizontalement option par défaut), 
					    % usepdftitle=info sur le titre et l'auteur
					    %brown couleur beamer pour les slides, par defaut c'est le bleu; autres: red, brown, blackandwhite

\usepackage[frenchb]{babel}
\usepackage[T1]{fontenc}
\usepackage[utf8]{inputenc}
\usepackage{xcolor}
%\hypersetup{pdfpagemode=FullScreen} %force le plein écran;déconseillé pour des raisons de chargement

%différentes manière de definir une couleur personnalisée
\definecolor{macouleur}{rgb}{0.20,0.43,0.09} % vert moyen
%\definecolor{macouleur}{RGB}{51,110,23} % vert moyen
%\definecolor{macouleur}{HTML}{336E17} % vert moyen; les lettres en majuscule dans le cas HTML
%\colorlet{macouleur}{black} %pour une couleur nommée (celles de base).


%definitions des thèmes

%thèmes sans barre de navigation

%\usetheme{default} 
%\usetheme{Pittsburgh}
%\usetheme{Boadilla}
%\usetheme{Rochester}
%\usetheme{Madrid} 
%\usetheme{AnnArbor}
%\usetheme{CambridgeUS}

%themes avec un arbre de navigation

%\usetheme{Montpellier} 
%\usetheme{Antibes}
%\usetheme{Juanlespins}

%thèmes avec sommaire latéral

%\usetheme{Goettingen} 
%\usetheme{Hannover}
%\usetheme{Marburg}
%\usetheme{Berkeley}
%\usetheme{Paloalto}

%thèmes avec minicadre de navigation

%\usetheme{Singapore}
%\usetheme{Szeged}
%\usetheme{Berlin}
%\usetheme{Ilmenau}
%\usetheme{Dresden}
%\usetheme{Darmstadt}
%\usetheme{Frankfurt}

%thèmes avec un sommaire et des sous sections

%\usetheme{Malmoe}
%\usetheme{Copenhagen}
%\usetheme{Luebeck}
\usetheme{Warsaw}


%couleurs des éléments beamer 
%\usecolortheme[named=macouleur]{structure} %couleur de la structure du thème bleu par défaut définie à macouleur
\setbeamercolor{structure}{fg=beamer@blendedblue} %d'où ce bleu par défaut
\setbeamercolor{normal text}{fg=black,bg=white} %valeur par defaut ; fg signifie foreground avant plan en français
\setbeamercolor{alerted text}{fg=red}
\setbeamercolor{example text}{fg=green!50!black}
\setbeamercolor{background canvas}{parent=normal text}


%\usefonttheme{nom du theme de police}
%\useinnertheme{nom du theme interne}
%\useoutertheme{nom du theme externe}

%definition d'un theme ici nommé clair de scheme pour les boîtes
\beamerboxesdeclarecolorscheme{clair}{cyan}{magenta}


%sommaire automatique 
\AtBeginSection[]%au debut de chaque section; \AtBeginSubsection[] existe
{
  \begin{frame}
  \frametitle{Sommaire automatique}
  \small \tableofcontents[currentsection, hideothersubsections]
  \end{frame} 
}

%informations sur le titre et l'auteur pour la page de garde
\title[Projet J2EE]{Classement des filles d'une école} %[titre pieddepage]{titre de la présentation}
\author{Ivan Tchomgue - Antoine Burie\\ Guilhem Marion - Victor Ripplinger} %auteur de la présentation
\institute{Enseeiht} %cadre ou lieu de présentation
\date{\today} %date de création du diaporama

\begin{document} %début de la présentation

\logo{\includegraphics[height=5mm]{../img/logon7.png}}%logo 
%\logo{\insertframenumber/\inserttotalframenumber}%logo

%page de garde
\begin{frame}
    \titlepage
\end{frame}

%sommaire
%\begin{frame}
    %\frametitle{Sommaire}
    %\tableofcontents %affiche tout
    %\tableofcontents[currentsection] % Indiquera la section en cours ainsi que sa sous section en cours. Le reste apparaitra en transparent.
    %\tableofcontents[currentsubsection] % Ne seront transparents que les sous sections n'étant pas en cours.
    %\tableofcontents[hideallsubsections] % N'affichera plus du tout les sous sections.
    %\tableofcontents[hideothersubsections] % Affichera tout, sauf les sous sections des sections n'étants pas en cours.
    %\tableofcontents[pausesections] % Fera une pause avant d'afficher le nom de la section suivante. Il faudra donc cliquer pour faire apparaitre la section n°2, puis re-cliquer pour faire apparaitre la section n°3, ...
    %\tableofcontents[pausesubsections] % Pareil qu'au dessus, sauf que là ce sera aussi pareil pour les sous sections, on affichera donc tout, un à un.
%\end{frame}

\logo{\includegraphics[height=5mm]{../img/logon7.png}}%insertion d'un logo sur les pages suivantes

\section{Verbatim et ses dérivées}	%une section englobant plusieurs frames
\subsection{Utilisation des verbatims} % une sous-section englobant plusieurs frames
    %debut d'une page
\begin{frame}[fragile, label=frame1] % fragile => utilisation du verbatim, label => référence 
\frametitle{Titre} %titre de la page
\framesubtitle{sous-titre} %sous-titre
\insertlogo logo %inserion du logo ds une page

\end{frame}
\end{document}%fin de la présentation
